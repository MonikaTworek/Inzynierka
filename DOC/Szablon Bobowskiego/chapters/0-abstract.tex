\chapter*{Abstract}
\thispagestyle{chapterBeginStyle}

\addcontentsline{toc}{chapter}{Abstract}
In this paper we go through the design process of the Credential System from theory to practice. 

The first part - theory, starts by informal introduction of Credential System concept, its functionality and stating requirements. Next we formalize the definitions and state the cryptographic assumptions and recall most important cryptopgraphic primitives. Then we introduce the theoretical backbone, the system construction, that is the construction proposed by Camenisch and Lysyanskaya in Anonymous Credential from Bilinear Maps, later improved by Słowik and Wszoła. At the end of first part present the improvement to the credential verification procedure, namely reduction of complexity of calculations of pairing comparisons by Bobowski and Słowik.

The second part - practice, is transition from mathematical world of cryptography into practical implementation. In form of the technical documentation we formalize the structure and the message exchange interface to propose a bilinear maps based credential system. We conclude the paper with the form of proof of concept implementation of credential system.

This thesis aims to be a first throughout look on the practical design of credential system. We try not to focus on the mathematical structures, but rather use them as the grounded intuition for the practical implementation.

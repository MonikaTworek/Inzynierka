\chapter{Conclusion}
\thispagestyle{chapterBeginStyle}

\section{Summary of Contribution}
We have gone thought the design of Credential System based on bilinear pairings. Starting from some intuitions and goals, though definition and protocols to the implementation guidelines and simple examples. On each of stages we have chosen the solutions that lead to the versatile system. We hope that we have shown the value of such system for the real-world solutions though the whole process and examples mentioned.

We based the solution on the work of Camenisch and Lysyanskaya \cite{anon-creds-cl04, signature-scheme-with-efficient-protocols, pseudonym-systems}. Focused on improvements introduced by Słowik and Wszoła \cite{slowik-efficient-cl-lrsw} and considered possible extensions \cite{complexity-reduction-bobowski, bulletproofs} and their interoperability with the basic system.

One of the motivations behind work on Credential System design was to start the discussion on the topic of practical applications of such system, and it's feasibility. We hope that this will start the conversations.


\section{Future Work}
The Credential System (CreS) proposed in this paper fulfills functionality goals of system stated in the beginning of the design process, yet we acknowledge that there are still areas where improvements should be made:

\begin{itemize}[label=$\circ$]
    \item \textbf{Range Proofs}
    
    Mentioned only as an extension, range proofs should be incorporated into the credential system specification.

    \item \textbf{Other Signature Schemes}
    
    It should be considered to analyze other signature schemes for credentials that fulfill system requirements. As the CL-LRSW is linear in attributes in size the improvements can be made in this area.
    
    \item \textbf{Better Integration with Domains and Pseudonyms}
    
    It should be considered to incorporate more features of pseudonymous system into CreS e.g. Pseudonymous Certificate System proposed by Słowik and Wszoła in \cite{slowik-efficient-cl-lrsw}. 
    
\end{itemize}